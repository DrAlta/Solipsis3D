\section{Conclusions and perspectives}
\label{sec:ending}
We  presented our vision for a decentralized architecture for
virtual environments.  Our solution is based on an n-dimensional
Voronoi-based \ptp network, called RayNet. This allows it to
distribute communication and computational cost among the various
nodes present in the virtual space. Our architecture enables the
decentralization of complex tasks related to physic-realistic
modeling. To achieve this, nodes preliminarily filter the set of
avatars and objects that may collide with their own and then evaluate
collisions and physics for a small set of entities. Our solution also
supports dynamic objects that may be picked up and dropped by avatars
or controlled by means of user-defined software.  Moreover, it enables
adaptive streaming of 3D models in a completely decentralized fashion
while adapting streamed data to avatars' viewpoints.  Also, the use
of an n-dimensional overlay allows us to extend the metaverse to
support social or semantic proximity between object or avatar nodes.
Access to the metaverse is made possible by a navigator that may run
as a stand-alone platform or embedded within a web page.  The
navigator exploits interactive texturing to enable the visualization
of Web 2.0 components in the virtual world and it integrates tools for
modeling new 3D contents. Moreover, it supports operation on
resource-scarce devices such as mobile phones. Our project team is
currently working on a fully open-source implementation of the \sol
architecture, and we hope that the community of users will help us
improve \sol and enable us to evaluate its effectiveness
in a world-wide testbed.
       


