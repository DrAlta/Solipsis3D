
The Metaverse concept, first described by Neal Stephenson in his
science fiction novel 'Snow Crash' published in 1992, and more
generally depicted in the whole cyberpunk writing movement, has deeply
influenced generations of virtual reality pioneers, artists, game
designers and nowadays virtual worlds enthusiasts. Over the last
sixteen years, the notion has evolved toward a synonym for virtual
world, loosing progressively the 'Universe' part of the concatenation
and the huge and infiniteness feeling emanating from it.

However, for us the Metaverse is a system of numerous, interconnected
virtual and typically user-generated worlds (or Metaworlds) all
accessible through a single-user interface. According to this strict
definition, the only Metaverse existing today is a prehistoric one:
the World Wide Web itself. Plenty of virtual worlds flourish these
days claiming they are the Metaverse, but they only are a part of it,
as websites are the leaves of the worldwide web tree. We need three
things to reach this cyberpunk authors' dream. First, a way to sustain
the incredible amount of data and MIPS involved, then a set of
protocols to provide interoperability and finally new tools to build
virtual worlds as easily as a traditional HTML page. These
requirements are the cornerstones of our project.

The paper is structured as follows. We first present a synthetic
overview of the Solipsis research project in
Section~\ref{sec:project-overview}, and an analysis of related work in
Section~\ref{sec:relwk}. Then we describe how we envision to manage
decentralized virtual worlds, describing in details our peer-to-peer
architecture, how we manage 3D-model sharing and physics computation,
and how to stream 3D contents in an adaptative way. Finally, in
Section~\ref{sec:navigator}, we present our navigator, which is the
user interface used to interact with the Solipsis Metaverse, while in
Section~\ref{sec:ending}, we conclude the paper and outline our future
directions.


\section{Project overview}
\label{sec:project-overview}
In a word, Solipsis is a platform for massively multi-participant and
user-generated virtual worlds. It relies on a peer-to-peer
architecture that makes scalability its main characteristic: the
universe may thus be inhabited by an unlimited number of participants. As
there is no central authority, the virtual universe is by definition
public and the inhabitants' freedom, as well as the world builders and
developers' imagination, are boundless.

\subsection{Expected results}

We seek to spark off the emergence of an unbounded public virtual
universe that is designed, created, run, but also potentially hosted,
by people throughout the world. Freedom can be spotted as the main
characteristic of this opensourced system (license GNU/GPL v2+)
because the virtual universe does not belong to any organization; it
belongs to all the users.

The major deliverable is a new network communication protocol adapted
to the strong constraints of self-produced environments, to massively
multi-user applications and to virtual reality, which make the system
more scalable as the resources its uses are those provided by its
users.

We also work on creating an ergonomic and user-friendly interface to
allow non-professional agents to easily create 3D scenes and contents
(declarative or automatic modeling, tagging..). Nevertheless, we do
not want to mark a break with the actual flat 2D web; rather, we aim
to start a smooth transition towards an immersive Internet making the
most of both 2D and 3D. As we will see in part �V, our navigator can
be embedded in a regular webpage - in-web world - or map regular
webpages as an interactive texture on any 3D surface - in-world web.

The last major expected result is a deep analysis on the behavior
(virtual social life, game, education, services..) of metaverse users
to give directions for future developments.
