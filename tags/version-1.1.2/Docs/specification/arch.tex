\section{Architecture}
The architecture of the protocol, depicted in Figure~\ref{}, outlines
three major components that provide the basis for the protocol's
operation: RayNet, a DHT and a cadastre service.

\paragraph{RayNet}
Nodes are organized in a three-dimensional overlay network based on
the Raynet protocol. The arrangement of nodes in this overlay is based
on the locations of the corresponding sites in the \sol world. In
particular, each node is responsible for a section of the world
centered around its own site and comprising a region of space around
it. In simple terms, two nodes are neighbors in the overlay if their
corresponding sites are neighboring in the \sol world\df{This is not
  strictly true with raynet, right?}. This, mapping between overlay
and 3D-world positions allows RayNet to provide efficient means to
address and communicate with the nodes responsible for the sites
encountered by avatars roaming in the \sol world.\df{This is currently
a repetition. EDIT.}

\paragraph{DHT}
The second component visible in the protocol architecture is the
Distributed Hash Table (DHT). While the RayNet overlay allows nodes to
access data by means of location information, the DHT allows them to
retrieve information about objects with a given identifier. This is
essential to access information regarding avatars that are not, by
definition, tied to a specific location in the \sol world. Moreover,
it is also useful to access information about sites when their
complete description has not yet been retrieved.\df{Using the DHT for
  sites is convenient but not essential. Let's make sure we know why
  we are doing this.}  The DHT associates each UID with the
corresponding object's descriptor and with a list of the nodes that
have cached a copy of the object's complete description as described
in Section~\ref{sec:stor-inform-about}.

\paragraph{Cadastre}
Finally, the last component highlighted in the figure is a
\emph{cadastre} service, responsible for managing the locations that
are available for new sites in the \sol world. In this version, we
assume a centralized version of the cadastre service, but we are
actively working\df{We are, aren't we?} on a fully distributed
implementation. Nodes may contact the service for to (i) obtain a list
of the available locations for new sites; and (ii) to obtain a permit
to create a new site with specified dimensions around a given 3D
location.\df{Some signature mechanism would be nice here... how about
  some P2P signing mechanism... is there any such a thing? How about
  using quorums for that? Does this make any sense?}

%%% Local Variables: 
%%% mode: latex
%%% TeX-master: "specification"
%%% End: 
