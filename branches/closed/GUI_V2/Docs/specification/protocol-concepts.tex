\documentclass{article}
\newcommand{\df}[1]{\footnote{\textbf{Davide: }#1}}
\newcommand{\inlinedf}[1]{\textit{\textbf{Davide:} #1}}
\usepackage{xspace}
\newcommand{\sol}{\emph{Solipsis}\xspace}
\begin{document}
This documents summarizes the concepts and ideas on the Solipsis
protocol, discussed in Emmanuelle's office on November 22, plus some
details that I thought about while writing it.

\section{Basic Definitions}


\subsection{Solipsis Objects}
The Solipsis world consists of a set of \emph{objects}. Each object is
associated with a unique identifier (UID) and may be classified into
groups: \emph{roaming} or \emph{static}\df{These were movable and
  unmovable}. Roaming objects may roam around the virtual world, and
interact with it. Static objects on the other hand are associated with
a fixed position in the virtual space. In this version of the
specification, we consider only one type of roaming and one type of
static objects, respectively, \emph{avatars} and \emph{sites}.



%\paragraph{Avatars}


%\paragraph{Sites}


\subsection{Solipsis Nodes}
A \sol \emph{node} is defined as a logical instance of the \sol
application, running on a physical host. Each object in the solipsis
world is associated with a unique node, referred to as its
\emph{owner}. Moreover, we require that each host be the owner of
\emph{at least one} site in the \sol world.  Finally, as for \sol
objects, each node is associated with a unique identifier (UID).

Objects may be described with several levels of details. In this
version of the specification, we concentrate on a very coarse-grained
subdivision between two such levels. Nonetheless, the protocol can
easily support several levels of details with only minor
modifications. The two levels we consider consist of a complete
description of the object and of a an object descriptor. The former
contains all the information necessary for the 3D visualization of an
object, plus any sounds, videos, or multimedia content associated with
it. From the point of view of this protocol, such a representation
merely consists of a binary object that should be provided to the
rendering component and of an associated version number. The object
descriptor, on the other hand, consist of the minimal information
required to display a rough representation of the object and includes
the object's location, its shape, its size, and its visibility
radius. The shape of the object is represented by an identifier that
selects one from a set of predefined shapes; while its visibility
radius identifies the distance from which the object should be visible
in the absence of obstacles and without any visual aids.



\section{Protocol Overview}
The \sol protocol is responsible for managing the information
regarding sites and avatars, for maintaining it persistent and for
making it available to \sol nodes, according to the requests of their
users. In the following we first describe the main components of our
protocol.  Then we discuss how they are employed to store information
regarding sites and avatars; and finally describe the main operations
of the protocol.

\subsection{Main Protocol Components}
\label{sec:main-prot-comp}
The architecture of the protocol, depicted in Figure~\ref{}, outlines
three major components that provide the basis for the protocol's
operation: RayNet, a DHT and a cadastre service.

\paragraph{RayNet}
Nodes are organized in a three-dimensional overlay network based on
the Raynet protocol. The arrangement of nodes in this overlay is based
on the locations of the corresponding sites in the \sol world. In
particular, each node is responsible for a section of the world
centered around its own site and comprising a region of space around
it. In simple terms, two nodes are neighbors in the overlay if their
corresponding sites are neighboring in the \sol world\df{This is not
  strictly true with raynet, right?}. This, mapping between overlay
and 3D-world positions allows RayNet to provide efficient means to
address and communicate with the nodes responsible for the sites
encountered by avatars roaming in the \sol world.

\paragraph{DHT}
The second component visible in the protocol architecture is the
Distributed Hash Table (DHT). While the RayNet overlay allows nodes to
access data by means of location information, the DHT allows them to
retrieve information about objects with a given identifier. This is
essential to access information regarding avatars that are not, by
definition, tied to a specific location in the \sol world. Moreover,
it is also useful to access information about sites when their
complete description has not yet been retrieved.\df{Using the DHT for
  sites is convenient but not essential. Let's make sure we know why
  we are doing this.}  The DHT associates each UID with the
corresponding object's descriptor and with a list of the nodes that
have cached a copy of the object's complete description as described
in Section~\ref{sec:stor-inform-about}.

\paragraph{Cadastre}
Finally, the last component highlighted in the figure is a
\emph{cadastre} service, responsible for managing the locations that
are available for new sites in the \sol world. In this version, we
assume a centralized version of the cadastre service, but we are
actively working\df{We are, aren't we?} on a fully distributed
implementation. Nodes may contact the service for to (i) obtain a list
of the available locations for new sites; and (ii) to obtain a permit
to create a new site with specified dimensions around a given 3D
location.\df{Some signature mechanism would be nice here... how about
  some P2P signing mechanism... is there any such a thing? How about
  using quorums for that? Does this make any sense?}

\subsection{Maintaining Information about Sites and Avatars}
\label{sec:stor-inform-about}
The \sol protocol provides an effective means to store information
about sites and avatars. The automatic replication of popular data
makes the protocol resilient to the disconnection or failure of \sol
nodes, and augments the availability of data by making it available
from several sources.

\paragraph{3D-Models and Object Descriptors}
As we mentioned above, we consider two levels of details in the
description of objects. The first is a complete description comprising
the object's 3D model and any necessary information to completely
render the object. The second is an \emph{object descriptor}
containing the object's identifier and a minimal set of information
that may be used for collision detection or to display a simple sketch
of the object. Specifically, a minimal node description should include
the following information. 

\begin{itemize}
\item  
location
\item
shape
\item
size
\item
visibility
\end{itemize}



A key feature of the \sol protocol is the use of the RayNet overlay to
organize peers in a topology that mimics the distribution of their
sites in the 3D world. Specifically, a network host is associated
with a nodes in the RayNet overlay characterized by virtual coordinates
that are the same as those of its site in the 3D world. If a host owns
multiple sites in the 3D world, then it appears in the RayNet overlay as
multiple nodes. Note that the presence of a host in multiple points on
the overlay does not consitute a problem for the small-world routing
strategies exploited by RayNet. On the other hand, it allows the overlay
to perform additional optimizations by taking shortcuts on the
physical network that would not be available otherwise. 

\paragraph{Maintaining a node's own cell}
Throughout its lifetime, each node maintains the necessary information
to manage the interaction with the avatars that are visiting its site
or the corresponding cell.  This includes the complete description of
the site\df{Do we want to replicate to neighbors when the site goes
  beyond the border of a cell.}  as well as a list of descriptors of
the objects that are visible from anywhere in the cell.

The list is built using either a reactive or a proactive protocol.
\df{describe the protocol}

The list is built through a simple protocol. Specifically, nodes
inform their neighbors whenever they detect a change 



periodically\df{or whenever they detect changes: this may be a nice
  point in the evaluation if we manage to characterize one or a few
  usage scenarios for the \sol world.} send a copy of their list to
their nerighbors in the RayNet overlay. Receiving nodes, process this
list and add to their own list any node that is also visible from
their own cells.  To manage the disappearance of objects from the
visible list, nodes tag each entry in the list with the hopcount from
the node that originated the 



ls
ll
 the time at
which the entry was last updated. Entries that are not refreshed after
a timeout $t_v$ are removed from the visible list.




node may start from the list objects that are currently in its own
cell. From this, it can compute which of the objects from the list are
visible from outside the cell, using the visibility attribute in the
object descriptors. The node may then forward this information to
neighboring nodes, which may then use it to update their to their own
list and possibly re-propagate the information with the next
update. \df{This gives an idea but a correct description. It needs
  rewriting.}


The complete description of a site may also be cached by other \sol
nodes as we describe in Section~\ref{}. 





\paragraph{A note on RayNet cells}
It is important to observe that the position of a node in the RayNet
overlay is solely determined by location of the corresponding site in
the virtual world and \emph{not} by the size of the site or by the
presence of multiple hosts at nearby locations.  As a result, it is
possible that the RayNet cell associated with a node comprises a region
that covers only part of the corresponding object in the 3D
world. This has no influence on the behaviour of the protocol for two
reasons. First, RayNet only considers the neighborhood relationship
between nodes and does not associate any size or weight properties
with overlay nodes. Second, according to the above descriptions \sol
nodes maintain information regarding their own site and the objects
visible from their own cell, regardless of its size.


 Start by describing arrangement of nodes in raynet


Each node always maintains complete information for all
the sites and avatars it owns, while the \emph{descriptor} is instead
recorded in the DHT, together with some bookkeeping information as we
describe in the following.

REWRITE HERE AND NEXT PAR, IT IS NOT CLEAR 
say that each node maintains
complete description of owned objs
node descriptor of objects visible from its cell (CELL MUST BE DEFINED
BEFORE THIS POINT)


Each node is the primary maintainer of any information associated with
its own site.  As a result, it is the only node that is entitled to
modify either the complete or the short description of the site. To
manage consistency in the presence of multiple copies, nodes mark
their complete descriptions and object descriptions with two version
numbers, one for each description, that is incremented whenever\ the
description is updated.


The owner of an object is the only node that is entitled to modify
either the complete or the short description of an object. As soon as
it makes such an update, the owner node also marks the modified data
with a fresh version number.  Nodes other than the owner can instead
access either description at any time in order to display the object;
moreover, they may cache the complete description and make it
available to other \sol nodes. Specifically, each node maintains a LRT
(Least Recently Used) cache in which it records a copy of all the
complete descriptions of the objects it displayed. An entry in the
cache is marked as the most recent whenever it is displayed by the
node or it is provided to another node that requested it. In order to
make the availability of such cached copies known, any node that
caches the complete description of an object records its identifier in
the DHT along with the description's version number.



\subsection{Protocol Operations}
\newcommand{\anode}{avatar node\xspace}
\newcommand{\snode}{site node\xspace}
Next, we describe the high-level operation of the protocol in response
to some relevant events that occur while managing the 3D world.
Throughout our description, we make use of the terms \anode and \snode
to refer to the two roles that may be played by a \sol node:
respectively,  being responsible for an avatar or for a site.

\paragraph{Joining Solipsis: Creation of a Site}
Users may join the \sol network through the creation of a new
site. Such a site is permanently associated with the node through
which it was created\footnote{Relocation of sites to new nodes and
  movement to different locations are not covered by this version of
  the specification.} and determines the location of the node in the
RayNet overlay. More precisely, a node joining the \sol network for
the first time should contact the \emph{cadastre} service to obtain a
new location for its site. Based on this location the node determines
its position in the RayNet overlay by computing the set of its
neighbors. From this moment on, the node becomes responsible for a
region of space containing, at least, its own site and some of the
space separating it from neighboring sites. In the following, we will
use the term \emph{cell} to refer to the region of space assigned to a
node by such process.


At each instant, each node maintains a complete description of the
sites included in its cell (these may include other nodes'
  sites).\df{We need to explain this better.}. In addition, it also
maintains a list of all the objects that are visible from its cell. To
build this list, a node may start from the list objects that are
currently in its own cell. From this, it can compute which of the
objects from the list are visible from outside the cell, using the
visibility attribute in the object descriptors. The node may then
forward this information to neighboring nodes, which may then use it
to update their to their own list and possibly re-propagate the
information with the next update. \df{This gives an idea but a
  correct description. It needs rewriting.}



As more nodes enter the network, their sites may occupy positions that
are close to those associated with previously existing sites. In this
case, RayNet automatically resizes the cells associated with existing
nodes so as to create a new cell that will be assigned to the joining
node. In doing this, it is possible that the approximate voronoy cell
associated with the owner of the smaller sites may end up containing a
portion of the larger site.  In this case, the owner of the smaller
site will be notified by the owner of the larger one about the
presence of the site.\df{Do we care? Possibly not, I need to think
  about it more thoroughly.}






% \inlinedf{Here we might have a problem. The approximate voronoy
%   tessellation performed by RayNet divides the space between two nodes
%   in a symmetrical way. This will probably create problems when a
%   small site is located close to a very large one. According to the
%   voronoy tessellation, a portion of the large site will, in fact, be
%   assigned to the node responsible for the small site. The issue is
%   that the mapping between regions and sites no longer holds. Whether
%   this issue is indeed a problem, however, is yet to be seen. Dividing
%   the management of a site among multiple nodes is certainly ugly, but
%   may also be justifiable. The question is where we should store the
%   information regarding the portion of a site that are not assigned to
%   the site creator. These questions remain unanswered in this version
%   of the document. }





\paragraph{Movement of an Avatar}
Similar to sites, avatars are also associated with the node that
created them. However, their location is not fixed and they can freely
roam around the virtual world visiting sites. As an avatar visits a
site, its node should be provided with all the necessary information
to display the avatar's current view of the site. This necessarily
includes all the multimedia content associated with the site as well
as any other displayable information associated with avatars that are
currently in it. In addition, the node should also be provided with
information regarding sites and avatars that are visible from the
avatar's position. In the following, we describe the details of the
protocol providing such information.


An \anode roaming around the 3D world constantly maintains contact
with the \snode responsible for its current location. This is achieved
by means of periodic beacon messages sent by the \anode to the \snode.
As soon as a \snode receives a beacon message from an avatar that was
not previously in its cell, it should provide the avatar with a list
of all the object visible from its current position. These include all
of the objects in its cell as well as visible objects in surrounding
cells. If the \snode has available resources, it may also provide the
\anode with a complete descriptions of all or some the sites in its
cell.\df{I put the direct download of descriptions as an optional step
  and I am not even sure we want to keep it. The problem is that an
  avatar that enters a cell might already have the descriptions of the
  objects in the cell, either because it retrieved them from farther
  away or because it has been moving in and out of the cell.} The
\anode can immediately start building a scene using these descriptions
while it downloads more descriptions.  To achieve this, it looks up
each of the remaining objects in the DHT and downloads its
descriptor. If a rough rendering is not sufficient\df{Here we need
  more input from rendering people.}, then it selects one of the nodes
that are listed as having a copy of the latest version of the object
and downloads the required data.

As side effect of an avatar's entering its cell, a \snode should also
update the list of objects in its cell and propagate the update to all
the avatars that may see the newly arrived avatar. To achieve this it
informs all the avatars that are currently in its cell. In addition,
it evaluates whether the new avatar will be visible from neighboring
cells and if so, it informs all the nodes responsible for those that
are within the visibility radius of the avatar. A simple way to
compute whether an avatar should be visible from neighboring cells is
to evaluate the presence of the new avatar as a variation in the
number of avatars already in the cell. If such a variation corresponds
to a very low percentage\df{need to define this}, then the change will
most likely go unnoticed and propagation can be avoided. If, on the
other hand, the percentage of change is greater than a threshold, then
the information about the new avatar is propagated to neighboring
cells.  When receiving such an update from a neighboring cell, a
\snode computes whether the update is going to affect the scene seen
by the avatars in its and in neighboring cells and propagates the
update as necessary.\df{Expand...}


\paragraph{Scalability and Fault Tolerance}
The peer-to-peer nature of the \sol protocol makes it easy to
guarantee that (i) information is likely to be retrieved even after
the disconnections of the nodes that were previously hosting it and
(ii) it can be downloaded efficiently even in the presence of a very
large number of requests.

To achieve both of these goals, we exploit replication when storing
information about objects in the 3D world. Specifically, each node may
cache the complete description of each site it displays and record
this fact in the DHT, along with the version number associated with
such information.  \inlinedf{ Here we need to determine whether we
  want to deal with partial views.  }  When downloading the complete
description of an object, a node may therefore choose among the nodes
that have an up-to-date replica, by examining the object's entry in
the DHT.  A possible metric for choosing the best node to download
from may be the average load on nodes that have the desired data.




\paragraph{Disconnection of a Node}
Replication is also the key mechanism enabling the \sol protocol to
maintain persistence in the event of a failure or disconnection of a
node associated with one or more sites. Addressing such an event
requires that some nodes replace the disconnected node in (i)
providing a complete description about the sites\df{What happens to
  the avatars?}  owned by the node; and (ii) managing the list of
avatars that are currently visiting the failed node's cell.

Goal (i), can easily be achieved by exploiting the replication
mechanism described above. First, we need to observe that the
disconnection of a node may cause an automatic reconfiguration in the
RayNet overlay, determining a change in the size of the cell
associated with a node.  As soon as a node detects such a change, it
uses the DHT to retrieve the list of sites owned by the disconnected
node. It should then go through this list to determine whether any of
the sites are now located in its own region. If this is the case, it
contacts one the nodes that are listed as having a copy of the sites
and downloads it.












\end{document}

