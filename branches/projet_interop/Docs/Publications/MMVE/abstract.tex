Lack of scalability is a key issue for virtual-environment technology,
and more generally for any large-scale online experience because it
prevents the emergence of a truly massive virtual-world infrastructure
(Metaverse). The Solipsis project tackles this issue through the use
of peer-to-peer technology, and makes it possible to build and manage
a world-scale Metaverse in a truly distributed manner. Following a
peer-to-peer scheme, entities collaborate to build up a common set of
virtual worlds. In this paper, we present a first draft of the \sol
architecture as well as the communication protocol used to share data
between peers. The protocol is based on Raynet, an n-dimensional
Voronoi-based overlay network. Its data-dissemination policy takes
advantage of the view-depedent representation of 3D
contents. Moreover, the protocol effectively distributes the execution
of computationally intensive tasks that are usually executed on the
server-side, such as collision detection and physics
computation. Finally, we also present our web component, a 3D
navigator that can easily run on terminals with scarce resources, and
that provides solutions for smooth transitions between 3D Web and Web
2.0.  

\textbf{Keywords:} Peer-to-peer System, Metaverse, Shared Virtual
Worlds, Massively Decentralized System, Adaptative 3D Streaming.
