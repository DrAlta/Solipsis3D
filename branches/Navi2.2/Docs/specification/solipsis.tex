\section{Solipsis Protocol Details}

\newcommand{\nd}{\ensuremath{\textsc{nd}}\xspace}
\newcommand{\UID}{\ensuremath{\textsc{UID}}\xspace}
\newcommand{\pos}{\ensuremath{\textsc{pos}}\xspace}
\newcommand{\sn}{\ensuremath{\textsc{sn}}\xspace}

\newcommand{\clist}{\ensuremath{\textsc{cList}}\xspace}
\newcommand{\descs}{\ensuremath{\textsc{descs}}\xspace}
\newcommand{\rn}{\ensuremath{\textsc{rNet}}\xspace}
\newcommand{\gn}{\ensuremath{\textsc{getNeighs}}\xspace}
\newcommand{\route}{\ensuremath{\textsc{route}}\xspace}
\newcommand{\cursn}{\ensuremath{\textsc{cursn}}\xspace}
\newcommand{\cupdate}{\ensuremath{\textsc{cUpd}}\xspace}
\newcommand{\aupdate}{\ensuremath{\textsc{aUpd}}\xspace}


\newcommand{\olist}{\ensuremath{\textsc{ownList}}\xspace}

\newcommand{\vn}{\ensuremath{\textsc{ver}}\xspace}

\newcommand{\msgentry}[1]{\vspace{2ex}\noindent{\large #1}\\}
\newcommand{\recby}[1]{\textbf{received by:} #1\\}
\newcommand{\sentby}[1]{\textbf{sent by:} #1\\}
\newcommand{\sentwhen}[1]{\textbf{when:} #1\\}
\newcommand{\param}[1]{\textbf{parameters:} #1\\}
\newcommand{\desc}[1]{\\}


\newcommand{\dstruct}[1]{\vspace{2ex}\noindent{\large #1}\\}
\newcommand{\mby}[1]{\textbf{maintained by:} #1\\}
\newcommand{\nnot}{\textbf{not}\xspace}




%data structures

%renderer 
\newcommand{\modtab}{\ensuremath{\textsc{3Dmodels}}\xspace}
%table containing an entry for each object for which there is at least
%a cached 3D model. 

%avatar node
%\newcommand{\vlist}{\ensuremath{\textsc{vList}}\xspace}
\newcommand{\snode}{\ensuremath{\textsc{sNode}}\xspace}

%site node
\newcommand{\vlist}{\ensuremath{\textsc{vList}}\xspace}

%own descriptor
\newcommand{\myDesc}{\ensuremath{\textsc{myDesc}}\xspace}
\newcommand{\loc}{\ensuremath{\textsf{loc}}\xspace}

%management of 3D descriptions
\newcommand{\modreq}{\ensuremath{\textsc{3Dreq}}\xspace}
\newcommand{\modrep}{\ensuremath{\textsc{3Dresp}}\xspace}


%management of views
\newcommand{\avhb}{\ensuremath{\textsc{avNodeHBeat}}\xspace}
%from avatar to site, periodic

\newcommand{\snhb}{\ensuremath{\textsc{siteNodeHBeat}}\xspace}
%from site to avatar, periodic

\newcommand{\vupd}{\ensuremath{\textsc{vUpdate}}\xspace}
%from site to site, periodic

\newcommand{\avupda}{\ensuremath{\textsc{avUpdate}}\xspace}
\newcommand{\avupdb}{\ensuremath{\textsc{avUpdate}}\xspace}
%from avatar to avatar, when avatar moves and with minimum frequency

\newcommand{\ncache}{\ensuremath{\textsc{newCopy}}\xspace}
%from avatar to avatar or site, when caching 3D description

\newcommand{\bRadius}{\ensuremath{R_b}}
\newcommand{\pRadius}{\ensuremath{R_p}}

%management of objects
\newcommand{\oreq}{\ensuremath{\textsc{objRec}}}
%from avatar to avatar or site, when requesting management of an object
\newcommand{\oack}{\ensuremath{\textsc{objAck}}}
%from avatar or site to avatar to grant management of an object
\newcommand{\oref}{\ensuremath{\textsc{objRef}}}
%from avatar or site to avatar to refuse management of an object
%%%%%%%%%%%


\newcommand{\safed}{\ensuremath{\textsc{safeD}}\xspace}
\newcommand{\neighd}{\ensuremath{\textsc{neighD}}\xspace}

%timers
\newcommand{\acqT}{\ensuremath{T_{\text{acq}}}\xspace}
\newcommand{\delTimer}{\ensuremath{T_{\text{del}}}\xspace}
\newcommand{\tavu}{\ensuremath{T_{\text{avu}}}\xspace}


\newcommand{\objUID}{\ensuremath{\textsc{objUID}}\xspace}
\newcommand{\objDesc}{\ensuremath{\textsc{objDesc}}\xspace}

\newcommand{\magNode}{\ensuremath{\textsc{magNode}}\xspace}
\newcommand{\reqNode}{\ensuremath{\textsc{reqNode}}\xspace}

%variables
\newcommand{\snhbRec}{\ensuremath{\textsc{snhbRec}}\xspace}
\newcommand{\avhbRec}{\ensuremath{\textsc{avhbRec}}\xspace}


%from avatar to site, periodic
\newcommand{\manager}{\ensuremath{\textsc{manager}}\xspace}
\newcommand{\seqNum}{\ensuremath{\textsc{seqNum}}\xspace}




\subsection{Constants}
The following is a list of constants that affect the protocol
operation. 

\begin{tabular}{|l|p{6cm}|}
  \hline
  \safed & distance between the bounding spheres of objects below which
  the corresponding nodes must update their positions at a high
  frequency.\\ \hline
  \neighd & distance within which objects should exchange position
  updates lazily, resulting in a lower update frequency.\\
  \hline
\end{tabular}

\subsection{Data Structures Maintained at Node}
Solipsis nodes maintain a set of data structures to manage the
interaction with the objects that constitute the 3D world. In the
following, we describe each of these structures for site nodes and for
avatar nodes, respectively.

%\subsubsection{Site Nodes}

\dstruct{\vlist} 
\mby{Site Node} 

Each site node maintains a visible list, \vlist. The \vlist is a table
containing an entry for each object that is visible from the current
site node's cell. Each entry in the list consists of an object
descriptor and of a timestamp indicating when the entry was last
updated.  Each object in the table may represent a site or an avatar
and may be inside or outside of the current node's cell. However, the
implementation may use separate tables for each of these sets of
object, as long as the protocol's semantics remain the same as
specified in this document.

% \paragraph{Changed List}
% The changed list \clist is a complement to the visible list described
% above and records the identifiers of the objects in the visible list
% whose descriptors have been recently updated (for more details see
% Section~\ref{}).

% % A more fine grained distinction is
% % provided by the \emph{visible site list}, the \emph{avatar list}



\dstruct{\vlist}
\mby{Avatar Node}

Similar to site nodes, avatar nodes also maintain a visible list,
\vlist, containing an entry for each object that is potentially
visible from the avatar's location. As in the case of site nodes, each
object in the table may represent a site or an avatar in any of the
surrounding cells, although the implementation may also use a separate
table for each set of objects.


\dstruct{\snode}
\mby{Avatar Node}

Each avatar maintains an $\snode$ pointer, that always refer to site
node that is currently responsible for the avatar's location.

\dstruct{\modtab}
\mby{Host}

Each host maintains a model table \modtab that is accessible by all
avatar and site nodes associated with the host. The table contains
references to local copies of 3D models cached by the current host. At
each instant the \modtab table must always contain references to the
3D model of sites and avatars permanently associated with the
hosts. In addition, it may contain references to remote avatars and
sites as required for visualization purposes or in order to achieve
persistency in the presence of failures.

\newcommand{\achange}{\ensuremath{\textsc{avChange}}\xspace}
\newcommand{\adesc}{\ensuremath{\textsc{avDesc}}\xspace}
\newcommand{\newsnode}{\ensuremath{\textsc{NewSiteNode}}\xspace}
\newcommand{\cursite}{\ensuremath{\textsc{curSite}}\xspace}


\subsection{RayNet}
Both avatar and site nodes have access to a \rn object that abstracts
their interface with the RayNet overlay network. The \rn object
provides each site node with with the following two functions.
\begin{itemize}
\item $\gn()$: may be called only by site nodes and returns the list
  of the node's neighbors.
\item 
$\route(m, p)$: may be called by both avatars and site nodes and routes a
message $m$ to the site node associated with the location $p$ in the
coordinate space. 
\end{itemize}



\subsection{Protocol Messages}
Protocol Messages are exchanged asynchronously between Solipsis nodes
using UDP. This may be overridden when the two communicating nodes (e.g. a
site and an avatar) actually reside on the same peer. 

% \paragraph{\achange}
% \recby{site node}
% \param{set of (name, value) pairs} 
% Inform the receiving site node of a change in one of the 


\subsubsection{View synchronization between avatar nodes}

\msgentry{\avupda}
\recby{avatar node}
\sentby{avatar node}w
\sentwhen{when moving, periodically with interval $T_{upd}$}
\param{ObjDescriptor nd}
\param{Set<Descriptor> descA, Set<Descriptor> descB}
\desc

Avatar update messages are exchanged between neighboring avatars to
maintain up-to-date information about close objects and particularly
about those that may interact with their movements. Each avatar node
should send an \avupda message whenever it changes its position or
other attributes such as parameters of its current
animation. Moreover, it should send additional \avupda messages to
prevent their frequency from falling below one update every $T_{upd}$
seconds. The contents of each \avupda message consists simply of the
object descriptor of the sending avatar.

Each avatar node computes the set of recipients of avatar update
messages based on its size and that of the avatar nodes in its cell as
specified in the node descriptors. Specifically, the set of recipients
includes all the avatar that are at a distance that is less than or
equal to the sum of their \bRadius values, plus the sender avatar's
\bRadius, plus a safe distance parameter $\safed$. This guarantees
that each update reaches all the avatars that may potentially collide
with the current avatar before the next update\footnote{During
  standard operation}.



Each \avupda message comprises two sets of descriptors. The first
always contains the node descriptor of the avatar sending and of any
object that is directly or indirectly attached to the
avatar. Information about the sending avatar and attached objects
contributes to implement the first level of the update mechanism
described in Section~\ref{sec:sysarch}. Specifically, it allows each
avatar node to obtain accurate information about objects that may
potentially collide with it in the near future. The second part of the
\avupda message is instead a list of node descriptors taken from the
sending avatar's \vlist. More precisely, this list includes
information about avatars (and attached objects) that are currently
closer to the sending avatar than to the recipient of the message and
that are at a distance that is less than $\neighd$ from the
latter. Note that each \avupda message does not need to contain all
the node descriptors for all the objects that satisfy the above
condition. On the other hand, each message only needs to contain
information about a randomly chosen subset of such objects. \df{To get
  better guarantees, we need a raynet-like structures for avatars as
  well.}  Still, if a message contains information about an avatar,
then it must also contain information about all of the objects that
are currently attached to it. This second part of the update message,
effectively implements the second level of the object update mechanism
described in section~\ref{sec:sysarch}. 

An avatar node reacts to the receipt of an \avupda by updating its
visible list. Specifically, it inserts or updates the entry for the
object descriptor listed in the message. If the \vlist contains no
entry for a given object, then corresponding descriptor from the
message is simply inserted in it. If, on the other hand, the \vlist
already contains entry for the object, the receiving node replaces it
with the received one only if the latter is associated with a fresher
sequence number.



\subsubsection{View synchronization between site and avatar/object nodes and
  between site nodes}
\msgentry{\avhb}
\recby{site node}
\param{ObjDescriptor nd}
\desc

Each avatar periodically sends an avatar heartbeat message to the site
node responsible for the avatar's current location, $\snode$. The
interval between messages, $T_{hb}$ should be on the order of
seconds. However, avatar nodes may occasionally send more frequent
updates, if their position or other features have undergone
significant changes. The receiving site node reacts to the an \avhb
message by updating its \vlist. Specifically, the entry associated
with the avatar is updated either if it was absent from the \vlist or
if the received entry has a fresher sequence number than the one in
the table.


\msgentry{\vupd}
\recby{site node, avatar node}
\param{list$<$ObjDescriptor$>$ nds}
\desc

A view update message contains the necessary information to update the
visible lists neighboring site nodes. The message is sent every $T_u$
seconds by every site node to each of its neighboring site nodes
($\rn.\gn()$). The contents of the message consist of a list of node
descriptors representing all the objects from the sender node's
visible list that are also visible from the sender node's current
position. A site node that receives a \vupd message from one of its
neighbors in the raynet overlay updates its visible list with all the
node descriptors in the message.

\msgentry{\snhb}
\recby{avatar node}
\param{list$<$ObjDescriptor$>$ nds}
\desc

A view update message contains the necessary information to update the
visible lists of the avatar nodes in a site node's cell. Similar to
\vupd messages, \snhb messages are sentevery $T_u$ seconds by each
site node to all of the avatars that are currently in their cells. In
this first version of the protocol sending should be done directly to
all avatars in the cell. Future versions will include a more efficient
data distribution strategy to reduce the forwarding load on site
nodes.  

Different from the case of \vupd, the contents of the a \snhb
message comprise all the node descriptors of the objects that are in
the sender site node's visible list. An avatar receiving the message,
uses it to update its visible list.  Specifically it adds to it all
node descriptors in nds which are not already present. In addition, it
updates each entry in the list for which there is a newer node
descriptor (with a fresher sequence number) in nds.

\emph{It should be noted that $T_{u}$ may be on the order of seconds
  or tens of seconds.}


\subsubsection{Managing Object Nodes}


\msgentry{\oreq}
\sentby{avatar node}
\recby{avatar node, site node, object node}
\sentwhen{when requesting management of object}
\param{UID objUID}
\desc

The message is sent by an avatar node whenever it detects that it has
collided with an object, or whenever the avatar intends to touch or
pick up the object. After the message has been sent the avatar node
may optimistically start managing the object but management should be
continued only if permission is granted by the previous
manager. Moreover, no updates about the object may be sent before
permission is granted. If no response to the message is received
within \acqT, the node assumes that the request has been lost and
sends a new request. 

A node receiving an \oreq message regarding a given object, with
identifier \objUID, looks for the object's descriptor in its \vlist
and verifies that its \magNode field is equal either to its own UID or
to that of the sender of the \oreq message.\footnote{The latter
  condition is necessary to tolerate a certain degree of message
  loss.}  \df{Add constraint for free/personal objects} If both
constraints are satisfied, the receiving node updates the \magNode
field in the descriptor to match the UID of the sender of the \oreq,
and then responds with an \oack message. Otherwise, if either of the
above conditions is not satisfied, the node replies by sending an
\oref message.

It should be noted that according to the above description, if new
requests are received for the same \objUID from the same \reqNode,
they are processed normally to provide some form of fault tolerance. 

\msgentry{\oack}
\sentby{avatar node, site node, object node}
\recby{avatar node}
\sentwhen{when granting management of object}
\param{Descriptor objDesc}
\desc 

The message is sent by an avatar node, site node, or object node
in response to an \oreq in order to grant permission to manage the
specified object. 

Upon receipt of this message, the destination avatar node updates the
corresponding descriptor in its \vlist and starts managing physics for
the object. 

\paragraph{Special processing carried out by object nodes}
After sending the message, an object node also starts a timer
\delTimer(\objUID), where \objUID=\objDesc.\UID. If a previous timer
was scheduled for the same \objUID, then the node first cancels the
previous timer and then starts a new one.  At the expiration of the
timer, the object node commits suicide and releases all resources.



\msgentry{\oref}
\sentby{avatar node, site node, object node}
\recby{avatar node}
\sentwhen{when refusing ownership of object}
\param{UID \objUID, UID \manager}
\desc

The message is sent by an avatar node, site node, or object node, to
refuse a request to manage the object with identifier \objUID. I a
node refuses such a request because it is not the node managing the
object, then it also records the identifier of the actual manager in
the message. 

An avatar node receiving the refusal message checks the value of the
\manager field. If the value is not null then it sends a new request
to the specified manager, otherwise it simply assumes that the
management of the object cannot be acquired. 


\subsubsection{Joining Operations}
\paragraph{Avatars}
An avatar node wishing to join the \sol needs two pieces of
information: a joining position, and the identifier, or the IP and
port, of at least one site node. The joining position should in
general be the last of the avatar in the \sol world but may also be
any other position, as in the case of an avatar joining for the first
time. The identifier or IP address and port number of a site node are
instead required to initialize the Raynet proxy so that it can
effectively route messages enabling the avatar to join the \sol
metaverse.

Once these two pieces of information are available, the actual joining
operation consists simply of initializing the Raynet proxy and routing
an \avhb message to the avatar's joining position.


\paragraph{Sites}
A site node wishing to join the \sol world\df{Here we need to add
  right management} needs, at a minimum, two pieces of information,
the desired position of its site and the identifier or IP address
and port of an existing site node that is currently in the \sol
world. 

Once the two pieces of information are available, the site node
initializes the RayNet layer by providing it with the information
regarding an existing site node in \sol. Then after a setup time
$T_{setup}$, it performs a join operation on the RayNet, thereby
joining the overlay. From this moment on, the node starts listening
for messages from avatars and neighboring site nodes and starts
disseminating the update messages as described above. 
 

\subsubsection{Persistence}
Persistence is the ability of the \sol metaverse to maintain
information about an entity while the corresponding host is
offline. This should be achieved both when hosts disconnect nicely,
possibly informing other peers, and when they disconnect abruptly
without any advance notice. In \sol, we consider two types of
persistency: one regarding sites and one regarding objects. Avatars,
on the other hand, disappear whenever the corresponding hosts
disconnect and reappear upon its reconnection.

\paragraph{Site Persistence}
Site persistence is achieved in \sol through replication and the
self-repairing capabilities of the Raynet overlay. 

\subsubsection{Managing 3D Descriptions}


\msgentry{\ncache}
%\sentby{host}
\recby{site node, avatar node}
\param{UID, \vn, $f$, url}
\desc

The \ncache message informs an avatar or a site node that a host has
cached a new copy of \vn version of the file $f$ associated with the
object identified by UID. 
The receiving node updates its private copy of the object descriptor
to include a reference to the new cached copy. \df{Need to agree on
  what to do with old versions}



\msgentry{\modreq}
\recby{host}
\desc

Hosts may obtain copies of 3D models by contacting one or more nodes
that have cached their latest version as specified in object
descriptors. To achieve this, they should send a \modreq message
specifying  the identifier of the object for which they are retrieving
the model, the type of model file they are retrieving (e.g. wireframe,
animation, ...), and the desired version number. 
The message is addressed to hosts regardless of whether they host site
or avatar nodes, and it triggers a response consisting of a \modrep
message as specified in the following.


\msgentry{\modrep}
\desc

A host receiving a \modreq message from another host should react by
looking up the desired 3D model in its own model table. If the table
contains a version of the required 3D model that is at least as recent
as the one requested by the \modreq, then it responds with a \modrep
message that includes a copy of such version. 




\section{Pseudocode}



%\begin{algorithm}
\begin{algorithmic}[1]

\Procedure{prepareHBRecipients}{}{prepares the list of recipients for
  site node heartbeats.}
%\State \snhb.\sn=\cursn++
\State assert(\snhbRec is empty)
\ForAll {$d \in \vlist$}
  \State \snhbRec.insert($d.\UID$)
\EndFor
\State $T_{hb}= T_u / \snhbRec.\text{size()}$
\EndProcedure

%\end{algorithmic}
%\vspace {2ex}
%\begin{algorithmic}[1]
\Procedure{sendSiteNodeHBeat}{}{Executed every $T_{hb}$ seconds by each site
node.}
\If {\snhbRec is empty}
   \State prepareHBRecipients()
\EndIf
\If {\nnot \snhbRec is empty}
   \State desc = \snhbRec.first()
   \State \snhbRec.erase(desc)
   \snhb.nds=\vlist
   \State send \snhb to desc 
\EndIf
\EndProcedure
%\end{algorithmic}

\Procedure{RecvSiteNodeHBeat}{\snhb}{}
\ForAll {$d \in \snhb.\text{nds}$}
   \State old=\vlist.get($d.\UID$)
   \If {$old = null \lor old.\seqNum \le d.\seqNum + 1$}
       \vlist.put($d.\UID$,$d$)
   \EndIf
\EndFor
\EndProcedure
%\end{algorithmic}


%\begin{algorithmic}[1]
\Procedure{sendVUpdate}{}{Executed every $T_{u}$ seconds by each site
node.}
 \State \vupd.nds=\vlist
 \ForAll {$n \in \rn.\gn$}
   \State send \vupd to $n$
 \EndFor
\EndProcedure
%\end{algorithmic}


\Procedure{RecvVUpdate}{\vupd}{}
\ForAll {$d \in \vupd.\text{nds}$}
   \State old=\vlist.get($d.\UID$)
   \If {$old = null \lor old.\seqNum \le d.\seqNum  +1 $}
       \vlist.put($d.\UID$,$d$)
   \EndIf
\EndFor
\EndProcedure

%\begin{algorithmic}[1]
 \Procedure{sendAvHeartBeat}{}{Executed every $T_{u}$ seconds by each
  avatar node.}
 \State \avhb.nd=\myDesc
 \rn.\route(\avhb, \myDesc.\loc)
\EndProcedure
%\end{algorithmic}


\Procedure{RecvAvHeartBeat}{\avhb}{}
   \State old=\vlist.get($\avhb.\text{nd}.\UID$)
   \If {$old = null \lor old.\seqNum \le \avhb.nd.\seqNum +1$}
       \vlist.put(\avhb.nd.\UID,\avhb.nd)
   \EndIf

\EndProcedure

\Procedure{sendAvUpdate}{}{Executed every \tavu seconds by each avatar
node.}

\If {\avhbRec is empty}
   \State prepareAvUpdRecipients()
\EndIf
\If {\nnot \avhbRec is empty}
 \State dest = \avhbRec.first()
 \State \avhbRec.erase(dest)

 \State \avupda.nd=\myDesc
 \State \avupda.desc=getDescriptorsWithin(dest.\loc\neighd)

 \State send \avupda to dest
\EndIf
\EndProcedure

\Procedure{RecvAvUpdate}{\avupda}{}
  \ForAll{$ d \in \avupda.\text{desc} \cup \{ \avupda.\text{nd} \}$}
    \State old=\vlist.get($d.\UID$)
    \If {old = null $\lor$ old.\seqNum $\le d$.nd.\seqNum$ +1 $}
      \vlist.put($d$.\UID,$d$)
    \EndIf
  \EndFor
\EndProcedure


\Procedure{prepareAvUpdRecipients}{}{prepares the list of recipients for
  avatar update messages.}
%\State \snhb.\sn=\cursn++
\State assert(\avhbRec is empty)
\ForAll {$d \in \vlist$}
  \If {mayCollide(d)}
     \State \avhbRec.insert($d.\UID$)
  \EndIf
\EndFor
\tavu = $T_{upd}$/\avhbRec.size()
\EndProcedure


\Function{mayCollide}{Descriptor d}{determines if an object is within
  potential collision distance}
\Return \myDesc.\loc.distance(d.\loc) - \myDesc.$R_b$ - d.$R_b$ -
\safed $\leq$ 0
\EndFunction


\Function{getDescriptorsWithin}{Coord \loc, double dist}{returns descriptors
  within d meters}
\State Set<Descriptor> \textsc{descs}
\ForAll {$d \in \vlist$}
  \If {\loc.distance(d.\loc) $\le$ dist}
     \textsc{descs}.insert(d)
  \EndIf
\EndFor
\Return \descs
\EndFunction

\end{algorithmic}

\begin{algorithmic}

\Procedure{sendOReq}{UID \objUID}{Executed when touching or picking up object.}
 \State Descriptor d=\vlist.get(\objUID)
 \If {\nnot d is null}
   %\If {\nnot d.pers}
   \State \oreq.\objUID=\objUID
   \State send \oreq to d.\magNode
  % \EndIf
 \EndIf
\EndProcedure
%\end{algorithmic}


\Procedure{recvOReq}{\oreq}{}
 \State Descriptor d=\vlist.get(\objUID)


 \If {old = null $\lor$ old.\seqNum $<$ \avhb.nd.\seqNum}
   \vlist.put(\avhb.nd.\UID,\avhb.nd)
 \EndIf

\EndProcedure

\end{algorithmic}





TODO:

- think about whether we absolutely need a DHT and what the advantages
and drawbacks of having it are. 

- add assumptions section

- define identifiers

- incorporate signing mechanisms for important messages
   - define important 
   - decide whether it is used
       - always
       - periodically
       - challenge-response

- update descriptor (or  what else???) with following information
about files:
   - hash of file
   - IP addresses of where to get it

- use DHT to get files when all else fails... (check)

- specify how to use DHT to map identifiers onto URL (make sure we
need it)










% % \State \snhb.\sn=\cursn++

% \State \snhb.\nds=\vlist

% \ForAll {$n \in \rn.\gn$}
%   \State send \cupdate to $n$
% \EndFor
% \ForAll {$d \in \olist$}
%   \State send \cupdate to $d.\UID$
% \EndFor
% \EndProcedure
% \end{algorithmic}
% \caption{site node sending \cupdate and \vupd messages}
% \label{alg:scvupdate}
% \end{algorithm}

% \begin{algorithm}
% \begin{algorithmic}





% \Procedure{sendSiteNodeHBeat}{}{Executed every $T_u$ seconds by each site
% node.}
% %\State \snhb.\sn=\cursn++
% \State \snhb.\nds=\vlist

% \ForAll {$n \in \rn.\gn$}
%   \State send \cupdate to $n$
% \EndFor
% \ForAll {$d \in \olist$}
%   \State send \cupdate to $d.\UID$
% \EndFor
% \EndProcedure
% \end{algorithmic}
% \caption{site node sending \cupdate and \vupd messages}
% \label{alg:scvupdate}
% \end{algorithm}








% %The pseudocode fragments corresponding
% %to these interpretations are given in Algorithm~\ref{alg:vupdate}



% % \begin{algorithm}
% % \begin{algorithmic}
% % \Procedure{sendViewUpdate}{}{Executed every $T_v$ seconds by each site
% % node.}
% % \State \vupd.\sn=\cursn++
% % \ForAll {$\nd \in \vlist$}
% %    \State \nd.\sn=\nd.\sn++
% %    \If {isInMyCell(\nd.\pos)}
% %       \olist[\nd.\UID]=\nd
% %    \EndIf
% % \EndFor
% % \State \vupd.\descs=\vlist
% % \ForAll {$n \in \rn.\gn$}
% %   \State send \vupd to $n$
% % \EndFor
% % \ForAll {$d \in \olist$}
% %   \State send \vupd to $d.\UID$
% % \EndFor
% % \EndProcedure

% % \Procedure{receiveViewUpdate}{\vupd msg}{Upon Receipt of a \vupd
% %   message by each avatar or site node.}
% % \ForAll {$\nd \in msg.\descs$}
% %   \If {\vlist contains \nd.\UID}
% %      \If {\vlist.get(\nd.\UID).\sn $$<$$ \nd.\sn}
% %         \State \vlist[\nd.\UID]=\nd
% %      \EndIf
% %   \Else
% %      \State \vlist[\nd.\UID]=\nd
% %   \EndIf
% % \EndFor
% % \EndProcedure




%%% Local Variables: 
%%% mode: latex
%%% TeX-master: "specification"
%%% End: 

