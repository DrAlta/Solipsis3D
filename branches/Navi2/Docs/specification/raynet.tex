\documentclass{article}
\newcommand{\df}[1]{\footnote{\textbf{Davide: }#1}}
\newcommand{\inlinedf}[1]{\textit{\textbf{Davide:} #1}}
\usepackage{xspace}
\usepackage{amsmath}
%\usepackage{algpseudocode}

\usepackage[latin1]{inputenc}
\usepackage[T1]{fontenc}
\usepackage{enumerate}
\usepackage[english]{babel}
\usepackage{amsfonts,amsmath,amssymb,amstext,latexsym}
\usepackage{boxedminipage}
\usepackage{subfigure}
\usepackage{rotating}
\usepackage{multicol}
\usepackage{floatflt}

\usepackage[vlined]{algorithm2e}
% \usepackage[vlined,boxed]{algorithm2e}
%\incmargin{1em}
%\linesnumbered
\dontprintsemicolon
\SetKwInOut{Input}{parameters}
\SetKwInOut{Output}{returns}

\usepackage{graphicx}
\usepackage {subfigure}
\newcommand{\sol}{\emph{Solipsis}\xspace}

\title{RayNet Specification}

\begin{document}
\newcommand{\toadd}[1]{\textbf{To Add: }#1}
\newcommand{\id}[1]{\ensuremath{\textsc{#1}}}
%\section{Introduction}

%%% Local Variables: 
%%% mode: latex
%%% TeX-master: "specification"
%%% End: 


%\section{Architecture}
The architecture of the protocol, depicted in Figure~\ref{}, outlines
three major components that provide the basis for the protocol's
operation: RayNet, a DHT and a cadastre service.

\paragraph{RayNet}
Nodes are organized in a three-dimensional overlay network based on
the Raynet protocol. The arrangement of nodes in this overlay is based
on the locations of the corresponding sites in the \sol world. In
particular, each node is responsible for a section of the world
centered around its own site and comprising a region of space around
it. In simple terms, two nodes are neighbors in the overlay if their
corresponding sites are neighboring in the \sol world\df{This is not
  strictly true with raynet, right?}. This, mapping between overlay
and 3D-world positions allows RayNet to provide efficient means to
address and communicate with the nodes responsible for the sites
encountered by avatars roaming in the \sol world.\df{This is currently
a repetition. EDIT.}

\paragraph{DHT}
The second component visible in the protocol architecture is the
Distributed Hash Table (DHT). While the RayNet overlay allows nodes to
access data by means of location information, the DHT allows them to
retrieve information about objects with a given identifier. This is
essential to access information regarding avatars that are not, by
definition, tied to a specific location in the \sol world. Moreover,
it is also useful to access information about sites when their
complete description has not yet been retrieved.\df{Using the DHT for
  sites is convenient but not essential. Let's make sure we know why
  we are doing this.}  The DHT associates each UID with the
corresponding object's descriptor and with a list of the nodes that
have cached a copy of the object's complete description as described
in Section~\ref{sec:stor-inform-about}.

\paragraph{Cadastre}
Finally, the last component highlighted in the figure is a
\emph{cadastre} service, responsible for managing the locations that
are available for new sites in the \sol world. In this version, we
assume a centralized version of the cadastre service, but we are
actively working\df{We are, aren't we?} on a fully distributed
implementation. Nodes may contact the service for to (i) obtain a list
of the available locations for new sites; and (ii) to obtain a permit
to create a new site with specified dimensions around a given 3D
location.\df{Some signature mechanism would be nice here... how about
  some P2P signing mechanism... is there any such a thing? How about
  using quorums for that? Does this make any sense?}

%%% Local Variables: 
%%% mode: latex
%%% TeX-master: "specification"
%%% End: 


%\section{Introduction}


\maketitle

This purpose of this document is to specify the behavior of the RayNet
overlay~\cite{}. Some basic design issues are also discussed, thus
providing a starting point for the implementation of RayNet in a
real-world setting. The document is structured as follows. 
 In Section~\ref{sec:gbps}
we present the general skeleton of gossip-based view maintenance
protocols. In Section~\ref{sec:rps} we present two such protocols that
constitute the basis of Raynet, and finally in Section~\ref{} we
present the architecture of Raynet and discuss the details of its
operation.




\section{Gossip-Based View Maintenance}
\label{sec:rps}
The behavior of RayNet follows the general scheme of a Gossip-Based
overlay maintenance service. Specifically, Raynet is based on the
combination of two gossip-based view-maintenance components that
contribute to building its fault-tolerant proximity-aware overlay
structure. In the following, we describe the general protocol
framework that lies at the basis of each of these view maintenance
protocols. 

A gossip-based view maintenance protocol, also referred to as
peer-sampling service, maintains an unstructured overlay network using
a gossiping procedure. Each peer maintains a data structure consisting
of a list of other peers, its neighbors. This list, normally referred
to as \emph{view}, is constantly refreshed using information exchanged
with other nodes in the overlay.

As described in~\cite{}, the skeleton of gossip-based peer sampling
protocols conforms to a simple general model and mainly differ in the
way in which the view is updated with each gossip exchange. The rest
of this section provides a description of this framework, while
Sections~\ref{sec:rps} and~\ref{} instantiate it into the building
blocks of RayNet. 

\paragraph{Peer-Sampling Framework}
In general, each network node is associated with an address (e.g. IP
and port) that is needed for sending a message to that node. In
addition it may also be associated with information needed in specific
peer-sampling instances. Each node maintains a membership table,
\emph{view}, representing its knowledge of the global membership.
This view is a list of $c$ \emph{node descriptors}.  Parameter $c$
represents the size of the list and is the same for all nodes.

A node descriptor contains the information associated to the node (as
defined above) and an \emph{age value} that represents the freshness
of the given node descriptor.  The partial view is a list data
structure, and accordingly, the usual list operations is defined on
it. The protocol also ensures that there is at most one descriptor
for the same node in every view.

The purpose of the gossiping algorithm, executed periodically on each
node and resulting in two peers exchanging their membership
information, is to update the partial views to reflect the dynamics of
the system.  Each node executes the same protocol, of which the
skeleton is shown in Algorithm~\ref{algo:prot}.  The protocol responds to
two events: a periodic timer event triggers the execution of the
active cycle initiating communication with other nodes, while the
receipt of a message invokes the  passive cycle responsible for
answering other nodes' requests.



\begin{algorithm}
\textbf{global variables:}\\
\textbf{set}[node] lastSent\\

\vspace{3ex}
\textbf{activeCycle}
\Begin{
  peer = select random peer from view\;
  message = create new message \;
  message.buffer = \id{selectToSend}(peer)\;
  m.sender=\textbf{self}\;
  lastSent=message.buffer\;
  \textbf{send} message \textbf{to} peer\;
}


\vspace{3ex}
\textbf{passiveCycle(Message m)}
\Begin{
  \If {\textbf{not} m \textbf{is} Response}{
    response = create new Response\;
    response.sender=\textbf{self}\;
    response.buffer = \id{selectToSend}(peer)\;
    lastSent=response.buffer\;
    \textbf{send} response \textbf{to} m.sender\;
  } 
  \id{selectToKeep}(m.buffer, lastSent)
}



\vspace{3ex}
\textbf{main active thread}
\Begin{
  \While{\textbf{true}}{
    activeCycle()\;
    sleep(\textbf{gossipInterval})\;
  }
}
\vspace{3ex}
\textbf{main passive thread}
\Begin{
  \While {\textbf{true}}{
    receive (message, sender)\;
    passiveCycle(message, sender)\;
  }
}


\caption{Skeleton of a gossip-based view-maintenance protocol}
\label{algo:prot}
\end{algorithm}


\paragraph{Active Cycle}

The active cycle is activated exactly once every $T$ time units.
First, it selects random node from the view  to exchange membership information
with.%   This selection is implemented by the method \id{selectPeer()}
% that returns the address of a node in the peer sampling's view.  This
% method is a parameter of the generic protocol. 

% Second, the peers exchange the membership information according to a
% communication strategy: either only one of the peer sends information
% to the other one (boolean parameter \id{push} is true if the active
% peer send its information, \id{pull} is true otherwise) or the two
% peers exchanged their information (\id{pushpull} is true).  

Second, the peer executing the active cycle exchanges membership
information with the selected peer by generating a gossip message. The
content of this message is determined by the
\id{selectToSend(recipient)} method. Choosing a specific
implementation of this method, allows each protocol instance to
determine the which peers to include in propagated messages,
ultimately controlling the characteristics of the resulting overlay.

Finally, at the end of the cycle, the age of each peer in the local
view is incremented by one. 


\paragraph{Passive Cycle}
The passive cycle reacts to the receipt of a message from another
peer. The message may either be a spontaneous message (i.e. sent in
the active cycle), or a response (sent in the passive cycle as
described in the following). In the first case, the protocol generates
a response message whose content is again determined by invoking the
\id{selectToSend(recipient)} method. 

After sending the message, or after determining that the received
message was already a response, the protocol proceeds to updating the
local view using the another protocol-specific method,
\id{selectToKeep(buffer, lastSent)}. The method takes as parameters
the list of node descriptors received with the gossip message being
processed, as well as the list of descriptors sent in the last gossip
message to the peer whose message is being processed. The first
parameter allows it to update the view with the new nodes; the second
allows the update mechanism to take into account the information
exchanged with the other peer. When removing descriptors from the view, it
is in fact beneficial to remove those that have been sent to another
peer, as this minimizes the overall loss of information.

Finally, as in the case of the active cycle, the age of each peer in
the local view is incremented by one.


\section{Instantiating the Peer Sampling}
The generic protocol described above is characterized by three
properties: namely, the selection of the peer to gossip with, the
selection of the information to be gossiped, and the computation of
the new membership table based on the exchanged information. In the
following, we present the two instances of the generic protocol that
constitute the raynet overlay.

\subsection{Kleinberg-Like Link Maintenance}
The first instance we consider is a mechanism to sample the nodes in
the network, so that each peer obtains a set of neighbors such that
the resulting overlay has the characteristics of a small world
network.  This is, in fact, at the basis of RayNet's ability to route
to any node in space in polylogarithmic time. In the following, we
present the protocol by describing the instantiation of the two
generic methods in the protocol framework.

% \paragraph{selectPeer}
% The selectPeer operation simply returns a random peer to gossip with
% from the currently available local view. 

\paragraph{selectToSend}
The \id{selectToSend(recipient)} operation is designed to return a set
of peers that are not immediately useful for the local peer's
view. This is convenient as these peers will be later removed by the
\id{selectToKeep} call. To achieve this biased peer selection, the
method first computes a temporary view $V'$ by removing the peer
selected by selectPeer (the destination of the gossip message) from
the local view.

Then each peer $p$ in $V'$ is associated with a probability of being
chosen $P_c(p)$ that depends on its distance from the local node. In
the context of \sol this is effectively the cartesian distance between
the locations of the site nodes. The probability follows a harmonic
distribution and is computed with the following formula where $d(n,
p)$ denotes the distance between the local node and the peer being
considered.

$$ P_c(p) = \frac{\frac{1}{d(n,p)^2}}{\sum_{p\in V'} \frac{1}{d(n,p)^2}} $$

After associating these probability with the peers in the view, the
function uses them to extract a set $K$ of $c/2$ peers that are of
interest to the current node. The remaining peers, that is $V' \setminus K$
are then returned to the generic protocol. 

\paragraph{selectToKeep}
The method \id{selectToKeep(buffer, lastSent)} modifies the local view
based on the received buffer with the objective to keep the most
interesting peer for the local node. To achieve this, it first appends
the received buffer to the view. Then it eliminates duplicate entries
from the view, and finally it removes items from the view starting
with the items contained in \id{lastSent} until the size of the view
is equal to $c$.


\subsection{RayNet Architecture}
The RayNet protocol can also be described as a gossip-based view
maintenance protocol. However, during its operation, it also exploits
information contained in the views of the kleinberg peer sampling
protocol described above. In the following we will refer the kleinberg
view as $V_K$, while the RayNet view consisting of the voronoy
neighbors will instead be referred to as $V_V$ In order to guarantee
successful routing, the size of $V_V$ is set equal to $c= 10$.

% \paragraph{selectPeer}
% As in the case of the previous protocols, the select peer operation
% returns a random peer from $V_V$. 

\subsubsection{selectToSend}
The method returns a random subset of $V_V$ containing $c/2$ nodes and
not containing the node selected by select peer. 

\subsubsection{selectToKeep}
The selectToKeep operation is the core of the RayNet
protocol. Different, from the previous protocol, in RayNet this
method evaluates the goodness of a view as whole, rather than
examining each peer individually.

The protocol operates by constructing a set $\id{rays}$ of $R$
\emph{rays} whose starting point is the local peer and whose and
directions are drawn uniformly at random on the unit hyper-sphere.
This is achieved using the method described
in~\cite{Knuth81The-Art-of-Computer-QF} that provides uniform
probability distribution of points on the hyper-sphere. More
precisely, the function $\id{createRays}(R)$ generates $R$ points
uniformly distributed on the surface of the sphere centered at the
local node, according to virtual world coordinates. Each ray is then
univocally determined by the one point in the sphere and the local
node.


 \begin{figure}[t]
  \centering
  \includegraphics[width=0.52\linewidth]{figs/schemas/rayons2d}
  \caption{Volume estimation using rays (o is the central point
    corresponding to the local node).}
  \label{fig.rayons2d}
\end{figure}


Rays (dashed lines starting from $o$ on Figure~\ref{fig.rayons2d})
will act as \emph{probes}, to discover the closest intersection point
$p_{int}$ lying on the ray $r$ with a (virtual) Vorono� cell of
another node in the configuration, this node being the node $n'$ for
which $\lambda = d(p_{int},n) = d(p_{int},n')$, where $n$ is the local
node, is minimal. This allows Algorithm~\ref{algo:volume} to compute
the estimated value of the voronoi cell associated with a set of
nodes. Specifically, in the algorithm, the function compDistOnRay() in
Algorithm~\ref{algo:volume} simply computes, for a pair consisting of
a ray and a node $n$, the point on the ray such that $\lambda =
d(p_{int},n) = d(p_{int},n')$ and returns the value of $\lambda$.



% Distances $\lambda = ||p_{int},n'||$ are
% represented by discontinuous lines from $n'$ to the intersection
% $p_{int}$ on Figure~\ref{fig.rayons2d}. 


\begin{algorithm}[h]


\textit{the set rays is essentially a set of points: set[Point] where
  Point x, represents a point x in the virtual world}

\textit{\textbf{calcVolume(set[nodes] config)}}\\
  \Begin{
        \textsc{double} $\Lambda[] \leftarrow \emptyset$ \;
    $\id{rays} \leftarrow $ \textit{createRays}(R)\;
    \lnlset{InResA}{(a)}\For{$r \in rays$}{
        double $\lambda \leftarrow \infty$ \;
        \For{node $n_j \in$ config}{
             double l $\leftarrow$ compDistOnRay($r$,$n_j$)\;
             \If{l < dist}{
                 $\lambda \leftarrow l$\;
             }
        }
        % lambda calculated at this point
        \lnlset{InResB}{(b)}$\Lambda \leftarrow \Lambda \cdot \lambda $ \;
    }
    % sum all lambdas
    \textit{/* BallVol contains the unit Ball volume in dimension $d$ */}\;
    \lnlset{InResC}{(c)}\Return{$\frac{BallVol \times \sum_{\lambda \in \Lambda} (\lambda^d)}{R}$}
  }



\caption{Monte-Carlo algorithm for estimating the volume of the cell for node~$n$~.}
\label{algo:volume}
\end{algorithm}

Lines~(a)~to~(b) of Algorithm~\ref{algo:volume} present the
selection of the closest peers for each ray. The function keeps all $\lambda$
values for each ray (set $\Lambda$), and uses them to compute the
estimation of the cell volume as follows (line~(c) of
Algorithm~\ref{algo:volume}).  Each ray $r$ is associated to a
\emph{ball} of radius $\lambda_r$ whose volume is given by $(BallVol
\times (\lambda_r)^{d}) / R$, where $BallVol$ is the volume of the
unit ball in dimension $d$.  The volume of the estimated cell is the
average value, for all rays, of volumes of such balls (the
contribution for each ray is represented as grey cones on
Figure~\ref{fig.rayons2d}). Such an estimator of the volume of the
Vorono� cell is clearly unbiased, so that the estimated volume
converges to the volume of the Vorono� cell when $R \rightarrow
+\infty$. Nevertheless, the convergence strongly depends on the shape
of the Vorono� cell, thus imposing the use of a large enough number of
rays ($10^3$).


\paragraph{Computing the best configuration}
The algorithm to compute the best configuration is shown in
Algorithm~\ref{algo:eff1}.  The algorithm maintains a bipartite graph
$best$ containing on one side the nodes in $V_V$, and on the other
side the rays $R$.  We denote by $best_n(r)$ the Vorono� neighbour
$n_r$ of $n$ according to ray $r$: it is the node $n_r$ such that a
ray issued from $n$ and whose direction is $r$ first reaches the
Vorono� cell of $n_r$ (this entry is never empty). Similarly, we
denote by $\{best_R(n^*)\}$ the set of rays for which $n^*$ is the
current Vorono� neighbour of $n$ (this set may be empty).

The objective is as follows: to compute the new view, for each node
$n_j$ in $buffer \setminus V_V$ (\textit{i.e.} all peers for which
$\{best_R(n_j)\}$ does not contain any information), we determine the
set of rays for which $n_j$ is the Vorono� neighbour of the local
node, $n$, in the augmented view Vorono� diagram.  This operation is
described by lines~(a)~to~(b) of Algorithm~\ref{algo:eff1}.  Peers
found to be a Vorono� neighbour of $n$ for a given ray are stored in
the set $improve$, which has the same semantic as $best$, except
that entries for some rays can be empty.

On line~(c), either $improve $ or $best$ has information, for each
ray, about which peer in the augmented view is a Vorono� neighbour of
$n$.  The next step is to compute to which extent each peer is needed
in the new configuration. More precisely, given a peer $n_x$, we
compute the volume of the cell of $n$ with all peers \emph{but} $n_x$
(lines~(c)-(d)).  If the volume of the cell increases dramatically,
that means that peer $n_x$ was mandatory to ensure closeness and
proximity.  On the other hand, if the volume remains the same, then
peer $n_x$ has no contribution to coverage nor closeness.

 
Volumes associated to each peer (\textit{i.e.} the volume without that
peer in the configuration) are stored in the map $volumes$.  This map
is then sorted by decreasing volume values~: starting from entries of
peers that contributes highly to coverage and closeness, to entries of
peers that have no or few contribution to coverage and closeness.  The
new configuration is built from the $c$ peers that presents the
maximum contribution, \textit{i.e.} peers of the first $c$ entries of
$volume$.

\begin{algorithm}[t]

\textit{\id{selectToKeep}(\textsc{set}[nodes] $buffer$,
  \textsc{set}[nodes] lastSent )}\\
/* lastSent is not used in this protocol */
  
  \textbf{Local variables:}\\
  ~~~~improve~(map ray $\rightarrow$ node) init $\emptyset$ /*improve has the same semantic as $best_n$*/\\
  ~~~~volumes~(list of pairs (node,volume)) init $\emptyset$\\
  \Begin{
    \lnlset{InResAb}{(a)} \ForEach{ray $r \in o.rays$}{
       double $best_\lambda = \perp$\;
       node $imp = \perp$\;

       \ForEach{node $n_j \in (buffer \cup V_K \setminus V_V)$}{
         $\lambda \leftarrow $ distOnRay($r,n_j$)\;
         \If{$\lambda < \left\{ \begin{array}{rcl} best_n(r) & \emph{if} & best_\lambda = \perp \\ best_\lambda & \emph{if} & best_\lambda \neq \perp \end{array}\right.$}{
           $imp \leftarrow n_j$\;
           $best_\lambda = \lambda$\;
         }
       }
       \If{$best_\lambda \neq \perp$}{
         % si on a trouve mieux
         \lnlset{InResBb}{(b)}improve[r] $= imp$ \;
       }
    }
    \lnlset{InResCb}{(c)}\ForEach{node $n_x \in V_V \cup (buffer \cup V_K
      \setminus V_V)$}{
       \lnlset{InResDb}{(d)}volumes $\leftarrow$ volumes $\cup$
       pair($n_x$,calcVolumeImproved($best \cup improve, (buffer \cup V_K
       \setminus V_V) \smallsetminus \{ n_x \}$))\;
    }
    sort volumes by decreasing volume\;
    $V_V \leftarrow \{volumes[1], \dots , volumes[c]\}$\;
    update $best_n$ and $best_R$\;
  }

%\vspace{2mm}
\caption{Update of node's view $V_V$. Sets $best_N$ and $best_R$ are constructed and coherent i.r.t. the current $V_V$ when starting the algorithm.}
\label{algo:eff1}
\end{algorithm}



\section{Initialization}
An important point when dealing with view maintenance protocols like
raynet, is the initialization of their views. In the case of RayNet
this process requires the insertion of a set of descriptors of known
nodes (at least one) in both the Kleinberg and the RayNet view, that
is in both $V_K$ and $V_V$. Both views may be initialized with the
same set.

\section{Routing in the RayNet}
The interface operation supported by the RayNet is \id{route(Message
  m, Point p)}
that routes a message to the node whose voronoi cell contains point
p. This is achieved by the simple steps shown in
Algorithm~\ref{algo:route}.


\begin{algorithm}[h]



\textit{\textbf{route(Message m)}}\\
  \Begin{
    \textbf{double} d=$\infty$\;
    \textbf{node} closest=\textbf{null}\;
    \textbf{Point} p=m.dest\;
        \ForEach{node $n_j \in V_K \cup V_V \cup \{n\}$}{
             \If {d > $d(n_j,p)$}{
               d=$d(n_j,p)$\;
               closest=$n_j$\;
             }
        }
        % lambda calculated at this point
        
    \If{closest=n}{
      \textbf{return} m to application
    } \Else {
      \textbf{send} m \textbf{to} closest
    }
    % sum all lambdas

  }
\vspace{3ex}
\textit{\textbf{receive(Message m)}}\\
  \Begin{
    route (m)
  }


\caption{Routing a message to a point.}
\label{algo:route}
\end{algorithm}









\end{document}








% on two types view exchange
% operations. The first constitutes the active phase of the RayNet
% protocol and is executed periodically by each peer. The second,
% implements its behavior triggered by the receipt of a message. The
% active phase allows peers in the RayNet protocol to initiate the
% dissemination of information about other peers in a gossip fashion.
% The passive phase, on the other hand, allows peers to respond to the
% messages sent by other peers in their view exchange operations. 

% \paragraph{Active Phase}
% The active phase of the RayNet protocol is executed periodically and 


% \paragraph{Passive Phase}
% The passive phase of the protocol allows \rn peers to process
% information received by other peers in the overlay and to use it to
% maintain their own sets of neighbors. 


% \section{Voronoi-cell estimation}
% The key characteristic of RayNet is the ability to select each peer's
% view so that it contains those that constitute its neighbor in the
% Delaunay graph. This is achieved by means of a simple algorithm that
% estimates the size of the voronoi-cell resulting from the choice of a
% given view. Each peer then seeks to maintain the view that minimizes
% the size of this cell so that its view eventually converges to its set
% of neighbors in the Voronoy/Delaunay graph. 



%%% Local Variables: 
%%% mode: latex
%%% TeX-master: t
%%% End: 
